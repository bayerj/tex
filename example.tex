% example document
% last updated 2016-06-16

\documentclass
[
%fontsize=11pt, % Document font size
twoside, % Shifts odd pages to the left for easier reading when printed, can be changed to oneside
%captions=tableheading,
%index=totoc,
%hyperref
]
{article}

\usepackage[l2tabu, orthodox]{nag}


% % % % % % % % % %
% MATH
% % % % % % % % % %
\usepackage[fleqn]{amsmath}	% fleqn for left alignment of math blocks (as in classicthesis)
\usepackage{amssymb}
\newcommand{\eq}[1]{\begin{align*}#1\end{align*}}
\newcommand\numberthis{\addtocounter{equation}{1}\tag{\theequation}}


% % % % % % % % % %
% FONTS
% % % % % % % % % %
\usepackage[osf,sc]{mathpazo} % Palatino as the main font
\linespread{1.05}\selectfont % Palatino needs some extra spacing, here 5% extra
%\usepackage[scaled=0.85]{beramono}
\usepackage[euler-digits]{eulervm} % nicer math font


% % % % % % % % % %
% FLOATS, FIGURAES, AND TABLES
% % % % % % % % % %
\usepackage{graphicx} 		% Required for including images
\usepackage{caption}
\captionsetup{font=small} 	% format=hang,
\usepackage{subfig}
\usepackage{tabularx}
\setlength{\extrarowheight}{3pt} % increase table row height
\usepackage{pbox} 			% for linebreaking in cells
\usepackage{multicol, multirow}
\usepackage{tikz}
\usetikzlibrary{fit,arrows,shapes}
\usetikzlibrary{decorations.pathreplacing}



% % % % % % % % % %
% BIBLIOGRAPY
% % % % % % % % % %


\usepackage[
	%backend=biber, %instead of bibtex
	backend=bibtex8,bibencoding=ascii,%
	language=auto,%
	style=numeric-comp,%
	%style=authoryear-comp, % Author 1999, 2010
	%bibstyle=authoryear,dashed=false, % dashed: substitute rep. author with ---
	sorting=nyt, % name, year, title
	maxbibnames=10, % default: 3, et al.
	%backref=true,%
	natbib=true % natbib compatibility mode (\citep and \citet still work)
]{biblatex}
\usepackage{biblatex}
% alternative: \usepackage{natbib}


% % % % % % % % % %
% MISC
% % % % % % % % % %
\usepackage
[
%disable
]{todonotes}
\usepackage[utf8]{inputenc} % allow utf-8 input
\usepackage[T1]{fontenc}    % use 8-bit T1 fonts
\usepackage[english]{babel} % hyphenation, special characters, ...
\usepackage{csquotes}		% recommended with babel for quotes
\usepackage{hyperref}       % hyperlinks
\usepackage{cleveref}		% provides \cref for nice in-document refs
\usepackage{url}            % simple URL typesetting
\usepackage{nicefrac}       % compact symbols for 1/2, etc.
\usepackage
[
activate={true,nocompatibility}, % activate protrusion and expansion
final,						% enable microtype; use "draft" to disable
tracking=true,
kerning=true,
spacing=true,
factor=1100,				% add 10% to the protrusion amount (default: 1000)
stretch=10,					% reduce stretchability (default: 20)
shrink=10					% reduce shrinkability (default: 20)
]
{microtype}
\usepackage{cancel}			% easy cancelling of terms: \cancel{expression}

\usepackage{msoelch}		% custom commmands, mostly math
\variables{x,W,y,z,u,v,w,a,b,c,r,C,d,A}
\probdists{p, q}

\title{Documentation for \texttt{msoelch.sty}}
\author{Maximilian Soelch}

\begin{document}
\maketitle

The \texttt{msoelch} package provides convenient and readable macros for commonly used syntax when writing scientific articles about machine learning.

\section{The \texttt{\textbackslash variables\{\}} macro}
A clear nomenclature is a necessity for good scientific writing. Macros can help a great deal to stick to one convention while maintaining readability of the source file. However, with an increasing number of conventions, maintenance of the macros becomes just as hard.

The \texttt{\textbackslash variables\{\}} macro eases some of that pain. It is called in the preamble with a list of variables and dynamically creates macros based on this list. For example, the call \texttt{\textbackslash variables\{x,y,A\}} automatically provides the following macros:

\begin{table}[hb]
	\centering
	\begin{tabular}{cccccccccc}
		\texttt{x} & \texttt{\textbackslash bx}& \texttt{\textbackslash xt} & \texttt{\textbackslash bxt} &\texttt{\textbackslash xT} & \texttt{\textbackslash bxT} & \texttt{\textbackslash xts} & \texttt{\textbackslash bxts} & \texttt{\textbackslash xTs} & \texttt{\textbackslash bxTs}\\
					& \bx & \xt & \bxt & \xT & \bxT & \xts  & \bxts & \xTs & \bxTs\\
		\texttt{y} & \texttt{\textbackslash by}& \texttt{\textbackslash yt} & \texttt{\textbackslash byt} &\texttt{\textbackslash yT} & \texttt{\textbackslash byT} & \texttt{\textbackslash yts} & \texttt{\textbackslash byts} & \texttt{\textbackslash yTs} & \texttt{\textbackslash byTs}\\
		& \by & \yt & \byt & \yT & \byT & \yts  & \byts & \yTs & \byTs\\
		\texttt{A} & \texttt{\textbackslash bA}& \texttt{\textbackslash At} & \texttt{\textbackslash bAt} &\texttt{\textbackslash AT} & \texttt{\textbackslash bAT} & \texttt{\textbackslash Ats} & \texttt{\textbackslash bAts} & \texttt{\textbackslash ATs} & \texttt{\textbackslash bATs}\\
		& \bA & \At & \bAt & \AT & \bAT & \Ats  & \bAts & \ATs & \bATs
	\end{tabular}
\end{table}

The driving idea behind these macros is the application to time series of scalars (normal font), vectors (bold font) and matrices (upper-case bold font). These macros can easily be edited and/or extended by adjusting the \texttt{msoelch.sty} file.

\section{The \texttt{\textbackslash probdists\{\}} macro}\label{sec:probdists}
The \texttt{\textbackslash probdists\{\}} macro provides a rich, overloaded macro for probability distributions.

The call \texttt{\textbackslash probdists\{p\}} provides a command that can be used in any of the following ways:

\begin{table}[hb]
	\centering
	\begin{tabular}{ccccccccc}
		\texttt{\textbackslash p}& \texttt{\textbackslash p\{x\}} & \texttt{\textbackslash p\{x\}\{z\}} &\texttt{\textbackslash p\{x\}\{z\}\{y\}} &\texttt{\textbackslash p\{x\}\{\}\{y\}} & \texttt{\textbackslash p\{\}\{\}\{y\}}\\
		\p & \p{x} & \p{x}{z} & \p{x}{z}{y} & \p{x}{}{y} & \p{}{}{y}
	\end{tabular}
\end{table}
The \texttt{\textbackslash p} macro dynamically decides which parts to display. This allows for rapid adjustment of the syntax. The parentheses adjust dynamically via internal usage of \texttt{\textbackslash mleft} and \texttt{\textbackslash mright}:
\eq{\p{\frac{x}{y}}}

Moreover, the \texttt{\textbackslash p} macro ignores any argument not given in set parentheses, \eg \texttt{\textbackslash p\{x\}\{z\}y} yields $\p{x}{z}y$.

Of course, \texttt{probdists} can be called with a list of letters, \eg \texttt{\textbackslash probdists\{p,q\}}.  At this point, more complicated constructions like $\mathbb{P}$ \etc are not supported since the letter is directly used for macro name construction.

\section{The \texttt{\textbackslash expc} and \texttt{\textbackslash var} macros}
For expectation and variance, this package provides overloaded macros similar to the dynamic macros provided by \texttt{\textbackslash probdists\{\}} in \cref{sec:probdists}.

The applicaion is straighforward:
\begin{table}[hb]
	\centering
	\begin{tabular}{cccc}
		\texttt{\textbackslash expc\{x\}} & \texttt{\textbackslash expc\{x\}\{z\}} &\texttt{\textbackslash expc\{x\}\{z\}\{y\}} &\texttt{\textbackslash expc\{x\}\{\}\{y\}}\\
		 \expc{x} & \expc{x}{z} & \expc{x}{z}{y} & \expc{x}{}{y} \\
		\texttt{\textbackslash var\{x\}} & \texttt{\textbackslash var\{x\}\{z\}} &\texttt{\textbackslash var\{x\}\{z\}\{y\}} &\texttt{\textbackslash var\{x\}\{\}\{y\}} \\
		\var{x} & \var{x}{z} & \var{x}{z}{y} & \var{x}{}{y}
	\end{tabular}
\end{table}

All comments from \cref{sec:probdists} apply here as well.

\section{Syntactic sugar}
\subsection{Math}
Some general, commonly used math expressions are provided by short, readable macros:
\begin{itemize}
	\item \texttt{mathbb} wrappers: \texttt{\textbackslash NN} for \NN, \texttt{\textbackslash RR} for \RR, \texttt{\textbackslash PP} for \PP, \texttt{\textbackslash EE} for \EE.
	\item \texttt{mathcal} wrappers: \texttt{\textbackslash mcX} for \mcX, \texttt{\textbackslash mcU} for \mcU, \texttt{\textbackslash mcZ} for \mcZ.
	\item \texttt{\textbackslash left\textbackslash right} wrappers: \eq{\texttt{\textbackslash abs\{\}}:\abs{\frac a b} \qquad\texttt{\textbackslash set\{\}}:\set{\frac ab}	\qquad\texttt{\textbackslash interval\{\}}:\interval{\frac ab}}
	\item Loss functions: \texttt{\textbackslash loss[index]\{arg\}} for $\loss[index]{arg}$. The parentheses adjust to the height of the arguments.
	\item Kullback-Leibler divergence: \texttt{\textbackslash kl\{p\}\{q\}} for $\kl{p}{q}$. The parentheses adjust to the height of the arguments.
	\item Gaussian/Normal distributions: \texttt{\textbackslash gauss\{0,1\}} for $\gauss{0, 1}$. The parentheses adjust to the height of the arguments.
	\item Integrals: \texttt{\textbackslash dint} for nicer integrals: \eq{\texttt{\textbackslash int x \textbackslash dint x}: \;\int x \dint x \qquad\quad \texttt{\textbackslash int x dx}:\;\int x dx}
	\item Equations that are to be proven: \texttt{x \textbackslash shallbe y} for $x \shallbe y$. 
\end{itemize}
\subsection{Other}
One of the most cumbersome typesetting issues is the correct spacing after abbreviations.

This packages provides macros that automatically apply correct spacing and follow \href{http://www.quickanddirtytips.com/education/grammar/ie-versus-eg?page=2}{$\rightarrow$this guide} for 
\begin{itemize}
	\item \emph{with respect to}: \texttt{derivative \textbackslash wrt input} $\leadsto$ derivative \wrt input,
	\item \emph{exempli gratia}: \texttt{algrorithms, \textbackslash eg EM} $\leadsto$ algorithms, \eg EM; also \texttt{\textbackslash Eg} for \Eg
	\item \emph{id est}: \texttt{156 members, \textbackslash ie almost all} $\leadsto$ 156 members, \ie almost all; also \texttt{\textbackslash Ie} for \Ie
	\item \emph{et cetera}: \texttt{Germany, France, \textbackslash etc} $\leadsto$ Germany, France, \etc
\end{itemize}
The latter command checks if a period follows, \ie \texttt{\textbackslash dots Germany, France, \textbackslash etc. Other countries \textbackslash dots} yields \dots Germany, France, \etc Other countries \dots

\section{Style of this document}
The default style of this document uses Palantino as the main font via \eq{\texttt{\textbackslash usepackage[osf,sc]\{mathpazo\}{}}\\\texttt{\textbackslash linespread\{{1.05}\}\textbackslash selectfont}} with some extra spacing between lines. Moreover, the Euler math font is used: \eq{\texttt{\textbackslash usepackage[euler-digits]\{{eulervm}\}}} The \texttt{amsmath} package is loaded with the \texttt{fleqn} for left flush of equations. 

All these changes were taken from the highly recommended \href{https://bitbucket.org/amiede/classicthesis/wiki/Home}{$\rightarrow$\texttt{classicthesis}} package by Andr\'e Miede.

Other recommended nice-to-have packages for convenient and good-looking typewriting are \texttt{cleveref} for extremely clever referencing, \texttt{nicefrac} for better handling of inline fractions, and \texttt{microtype}, which takes care of small typesetting issues that improve the overall document appearance.

\section{\TeX studio autocomplete}
Along with this style file and example document, a \texttt{cwl} file is provided. By adding it to your \TeX studio, autocomplete for the static macros is provided. At this point, autocomplete of the dynamically produced \texttt{variables} and \texttt{probdists} is not included.

Under Windows, it has to be copied to the directory \eq{\texttt{\%appdata\%\textbackslash texstudio\textbackslash completion\textbackslash user}.} 



\end{document}